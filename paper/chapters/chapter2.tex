\chapter{Technologie wykorzystane w projekcie}

W drugim rodziale przedstawiono technologie wykorzystane w implementacji programu. Głównymi czynnikami wyboru danych technologii była łatwość użycia oraz ich możliwości.

Najważniejszymy punktami wybranych technologii są m.in.:
\begin{itemize}
	\item Otwarty kod źródłowy
	\item Popularność
	\item Ogólnodostępność
	\item Aktywna społeczność rozwijająca projekty
	\item Łatwość wykorzystania
	\item Dostępności informacji dot. technologii
\end{itemize}

Dzięki powyższym cechom wykazywanym przez użyte technologie mamy zapewnioną szybkość tworzenia oprogramowania, gdzie w przypadku napotkania błędów w procesie jego kreacji możemy liczyć na pomoc społeczności w jego rozwiązaniu jak i szeroką gamę wcześniej stworzonych narzędzi.



\section{Python 3}
Sercem każdego programu jest język, w którym został on napisany. Językiem programowania, wybranym do implementacji projektu jest \textbf{Python} w wersji 3.11.
Jest to wysokopoziomowy, wieloparadygmatowy język skryptowy ogólnego przeznaczenia, którego ideą przewodnią jest czytelność uzyskanego kodu źródłowego. Powstał w 1991 roku i jest ciągle rozwijany oraz cieszy się od wielu lat niesłabnącą popularnością, która wynika między innymi z:

\begin{itemize}
	\item Łatwość nauki
	\item Dostępności materiałów dydaktycznych
	\item Wszechstronności języka
	\item Mnogości różnorodnych bibliotek
	\item Wieloplatformowość
\end{itemize}

\section{PyTorch}
Otwartoźródłowa biblioteka uczenia maszynowego stworzona przez Facebook AI Research (teraz Meta AI), bazowana na Pythonie i Torch-u, głównie używana do aplikacji uczenia maszynowego używających procesorów graficznych i TPU. PyTorch jest preferowany przez wielu badaczy sztucznej inteligencji przez używanie dynamicznych grafów obliczeń, co pozwala na uruchamianie i testowanie małych części kodu w czasie rzeczywistym bez potrzeby implementacji całości kodu, móc sprawdzić czy dana część kodu działa czy nie.

Głównymi zaletami PyTorch są:
\begin{itemize}
	\item Obliczenia tensorowe z dobrym wsparciem przyśpieszenia sprzętowego GPU i TPU
	\item Automatyczne różnicowanie dla tworzenia i trenowania głębokich sieci neuronowych
\end{itemize}

\section{OpenCV}
OpenCV (Open Source Computer Vision Library) jest otwartoźródłową biblioteką widzenia maszynowego oraz uczenia maszynowego. Została ona zbudowana by dostarczyć wspólną infrastrukturę dla aplikacji rozpoznawania obrazów oraz przyśpieszenia użycia percepcji maszynowej w produktach komercyjnych.

\section{Pillow}
Biblioteka będąca odgałęzieniem Python Imaging Library (PIL), która to dodaje możliwości przetwarzania obrazów do programów pisanych w języku Python. Dodaje ona szerokie wsparcie formatów plików, wydajną wewnętrzną reprezentację oraz duże możliwości przetwarzania obrazów. 

\section{Verovio Toolkit}
Verovio jest szybką, przenośną i lekką biblioteką do grawerowania cyfrowych zapisów nutowych MEI (Musical Encoding Initiative) do obrazów formatu SVG. Zawiera również konwertery do renderowania cyfrowych zapisów nutowych formatów MusicXML, Musedata, EsAC czy Humdrum.