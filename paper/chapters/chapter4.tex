\chapter{Użyte narzędzia programistyczne}
Rozdział ten zawiera opis narzędzi programistycznych wykorzystywanych podczas pracy nad projektem. Ich dobór, jak w każdej pracy twórczej, jest ważny dla ułatwienia procesu kreacji, dając większe możliwości programiście. Mogą one przyśpieszyć jego pracę, pozwolić na swobodne prototypowanie bez potrzeby naruszania właściwego kodu aplikacji, czy pozwolić na zarządzanie zdalnymi maszynami wykonującymi niezbędne obliczenia bez potrzeby przebywania fizycznie w tym samym miejscu, czy nawet posiadania takiej maszyny na własność. Głównym powodem wybrania przedstawionych niżej narzędzi było ich rozpowszechnienie, dopracowanie oraz darmowa licencja użytkownika.

\section{Git}
Jedno z najbardziej powszechnych i podstawowych narzędzi programistycznych, umożliwiające rozproszoną kontrolę wersji plików, która w przeciwieństwie do wielu alternatyw odbywa się lokalnie, gdyż każda kopia plików źródłowych kontrolowanych przez Gita może być pełnoprawnym repozytorium zawierającym całą historię zmian zachodzących na tychże plikach. Pozwala to na powrót do konkretnych wersji, których programista może potrzebować. Narzędzie to jest szeroko wykorzystywane do rozproszonego tworzenia oprogramowania przez grupy programistów w ramach jednego projektu, gdzie zmiany wprowadzane przez różnych ludzi są scalane do pojedynczego źródła, wykorzystując komercyjne rozwiązania hostowania online do przechowywania historii zmian.

Głównymi założeniami Gita jest:
\begin{itemize}
	\item Szybkość działania
	\item Elastyczność
	\item Bezpieczeństwo
	\item Wsparcie nielinowych, rozproszonych przepływów pracy
	\item Integralność danych
\end{itemize}

\section{Visual Studio Code}
Darmowy, otwartoźródłowy program do edycji, analizy i zarządzania kodem, stworzony i rozwijany przez Microsoft. Edytor ten jest z założenia edytorem minimalistycznym, zawierającym w swej podstawowej formie tylko niezbędne narzędzia, jednakże przez bycie projektem o otwartym źródle, jest on łatwo rozszerzalny dzięki szerokiej palecie wtyczek tworzonych przez podmioty komercyjne oraz członków społeczności. Szeroka gama rozszerzeń pozwala użytkownikom Visual Studio Code na dużą swobodę dostosowywania tego edytora pod swoje własne preferencje ergonomii pracy, jak i różnorodne projekty w większości języków programowania, przekształcając go w pełnoprawne interaktywne środowisko programistyczne, posiadające możliwość pracy ze zdalnymi zasobami.

Visual Studio Code posiada wbudowaną obsługę funkcji edycji kodu dla języków programowania \textit{JavaScript} oraz \textit{TypeScript}, gdzie przy pomocy wtyczek możliwa jest obsługa innych popularnych języków takich jak:

\begin{itemize}
	\item C/C++
	\item Python
	\item Java
	\item C\#
	\item Rust
	\item Go
\end{itemize}

Innymi ważnymi funkcjami tego programu są  m.in.:

\begin{itemize}
	\item Autouzupełnianie kodu
	\item Informacje dotyczące parametrów 
	\item Refactoring
	\item Analiza kodu
\end{itemize}
które to pozwalają na szybszą i bardziej efektywną pracę.

\section{SSH}
SSH, znane również jako Secure Shell, jest metodą bezpiecznego przesyłania poleceń do komputerów przez niezabezpieczoną sieć. Używa metod kryptograficznych do autoryzacji i szyfrowania połączeń pomiędzy urządzeniami. Pozwala również na tunelowanie czy przekierowywanie portów. Jest jedną z najpowszechniejszych metod zarządzania serwerami, infrastrukturą sieciową i przesyłania plików.